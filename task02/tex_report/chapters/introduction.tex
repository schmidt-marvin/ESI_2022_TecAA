\section{Introduction}
\label{CHAPTER_Introduction}
The task to be completed within this assignment is the prediction of damages dealt to buildings in Nepal during the 2015 Gorkha earthquake. To do this, we use an evolutionary approach for supervised learning techniques, in which a baseline with basic techniques is established first and improved using more advanced optimization techniques. 

The assignment is based on the competition "\textbf{Richter's Predictor: Modelling Earthquake Damage}” held by DrivenData.org. The dataset we are being provided consists of 39 features, from which the feature \texttt{buildingid} uniquely identifies a specific building in the region. Among various other descriptors of each building, the features, for instance, specify the geographical region or the building materials used. The label to be predicted, \texttt{damagegrade}, represents the level of damage dealt during the earthquake.

The competition provides a training dataset, a testing dataset and a submission template which can be used to create a submission file. The submission file can then be uploaded to the website to produce a score of the achieved submissions.