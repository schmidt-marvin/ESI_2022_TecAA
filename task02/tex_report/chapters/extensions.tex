\section{Extensions}
\label{SECTION_Extensions}

\subsection{Outlier Feature Clustering}
\label{SUBSECTION_OutlierFeatureClustering}

Another great idea could be to add new features using Laplace’s rule\footnote{Refer to: \texttt{Task02\_SupervisedLearning\_OutlierFeatureDetection\_Barba\_Guerrero\_Schmidt.ipynb}.}. In our case, we will only use these features obtained from conditional probabilities, each of them representing the probability of each earthquake to present a damage grade of 1, 2 or 3, given its geo level 1, 2 and 3, as shown in figure \ref{PICTURE_tableoffeatures_ofc}.

\begin{figure}[h!]
	\centering
            \includegraphicscentered{images/ofc_figure01.png}{130mm}
	\caption{Outlier Feature Clustering: Table of features.}
	\label{PICTURE_tableoffeatures_ofc}
\end{figure}

Since we have 9 features now, we can apply PCA to do dimensionality reduction as it will save us time and computational cost. Choosing 2 components explains 87\% of the total variance. By using k-means++ and trial and error, we decided to use 6 different components to cluster all the records in the dataset, providing us with the result seen in figure \ref{PICTURE_visualization_ofc}.

\begin{figure}[h!]
	\centering
            \includegraphicscentered{images/ofc_figure02.png}{130mm}
	\caption{Outlier Feature Clustering: Visualization.}
	\label{PICTURE_visualization_ofc}
\end{figure}

Finally, by using different boosting algorithms previously mentioned, we can obtain accuracy values of up to 0.74, which are slightly worse than the previous ones, but still very good. However, there is a final strategy to try to reach the top spots in the ranking.

\subsection{Two-phase classification stacked model}
\label{SUBSECTION_TwoPhaseClassificationStackedModel}

Finally, our last idea was to divide the classification part into two differentiated phases, since there are more samples from damage grade of 2 than from the other two \footnote{Refer to: \texttt{Task02\_SupervisedLearning\_TwoPhaseClassificationStackedModel\_Barba\_Guerrero\_Schmidt.ipynb}.}. Thus, our aim is to first classify each record into one of the two following classes: damage grade of 2 or class 0 (containing all records from the other damage grade records).

After this first classification, if we do not include the record into class 2, we must classify it into damage grade of 1 or damage grade of 3. This strategy prevents us from one of the main problems we used to have in the baseline: the model wrongly classifying records from the third class into the second one.

For both phases, we will use a stacked model with both a CatBoost model and an XGBoost model, since including LightGBM here does not help us improve the results. This implementation will provide us a confusion matrix shown in figure \ref{PICTURE_confmatrix_ofc} for the first and second phase, respectively.

\begin{figure}[h!]
	\centering
            \includegraphicscentered{images/2pcsm_figure01}{110mm}
	\caption{Two-phase classification stacked model: Confusion Matrices for both phases.}
	\label{PICTURE_confmatrix_ofc}
\end{figure}

This model has shown to be the best out of the ones tried, with a value of 0.7438, meaning a rank of 459\footnote{Rank 459, as of 13.12.2022, uploaded by user \texttt{RaulBarbaRojas} on 2022-12-11 19:48:05.}.